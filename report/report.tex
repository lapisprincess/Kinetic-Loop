\documentclass[sigconf,nonacm]{acmart}

\begin{document}

\title{Roguelike Development from Scratch}
\author{Matilda Lerner}
\date{\today}
\maketitle

\pagenumbering{roman}
\tableofcontents
\newpage
\pagenumbering{arabic}

Here's my random garbage <3

\newpage
\section{Layout}

\subsection{Abstract}
A short (max 150 words) summary of the paper. It’s usually easier to write this last, once the structure of the paper has taken shape.

\subsection{Introduction}
Explain the context and motivation for your project. What domain are you working in? What is the subject of your project? Why is this a problem worth solving? Include a very brief description of your approach. With some luck, you can probably steal most of this from your project proposal.

\subsection{Background and Related Work}
What other work has been done before? Do similar systems or research exist? How does your project fit into a greater context, perhaps building on what has come before? Be sure to reference other work properly, and include those references at the end of the paper.

\subsection{Implementation / Methods}
If you did an implementation project, describe its architecture and how it works. For a research project, explain your research method. Feel free to document failed attempts as well as the final path taken. This is likely to be the largest subsection in the writeup of an implementation project. Just the facts please: this subsection should not make claims about how well your system works or how it compares to other efforts. That material goes in the next two subsections.

\subsection{Evaluation / Results and Analysis}
How well does your system work, or what did you figure out? In an implementation project, you might report on either a short user study explaining the system in use, or give the results of tests demonstrating the effectiveness of your implementation. For research projects, present and analyze data that supports your arguments. It wouldn’t be surprising for this to be the largest subsection in a research project writeup.

\subsection{Discussion}
The previous two subsections described your work in detail, and the related work subsection talked about what others have done. This subsection should put your results in context — how does your contribution support, refute, or extend previous work? What is the significance of your project? Opinions are fine here, as long as you make a coherent argument to back them up (e.g. “We feel that this system offers a superior experience for the user since it crashes less often than Microsoft Blurb.”). Your discussion will be stronger if you give an honest comparison of your project to existing work (e.g. “It is true that Microsoft Blurb offers better security than our system, but security was not our primary emphasis for this proof-of-concept implementation, and could be added at later date.”).

\subsection{Future Work}
What are the next steps, either for you or for people who might build on your work?

\subsection{Conclusion}
Include some concluding remarks. Summarize and reiterate what you see as the key points from your writeup. Basically you’re using the old “tell them what you’re going to tell them, tell them, and then tell them what you told them” structure in the hopes that your point really got through.

\subsection{Acnowledgments}
This subsection is used to acknowledge the people or organizations that provided significant help or inspiration. It might not be relevant to your writeup. In the real world you’d thank the funding agencies that supported your work, the reviewers who gave feedback on drafts of the paper, and any other contributors. If you did an implementation project, you might consider acknowledging the authors of any packages or libraries you used, etc.

\subsection{References}
A properly formatted list of references. This will be substantial in a research report, but less so for an implementation project. The references can be construed fairly broadly in computer science—in addition to journal articles, books, and conference proceedings, papers often reference web pages, tutorials, etc. Note that this is different from a bibliography, which lists things that you’ve read but that are not necessarily referenced in your paper. All of the entries in this subsection should be referenced at least once in the body of your paper.

\subsection{Appendices}
You might consider adding an appendix (or several) if you have tables of data, long listings of source code, or other supporting information that’s too long to insert into the body of the paper, but that’s relevant to your project. These will not be counted against the 8-10 page limit.


\end{document}
